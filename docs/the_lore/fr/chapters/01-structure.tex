\chapter{Structure \& Directives}

\section{Métadonnées}
\textit{Définir la version, la date et le hachage unique de cette itération de l'histoire. Cela établit le "build" de l'univers.}

\section{Note de l'Auteur}
\textit{Un message personnel du créateur. Intentions, avertissements sur les thèmes ou la philosophie derrière la conception.}

\section{Guide de Lecture}
\textit{Instructions sur la façon de naviguer dans ce document. Est-ce linéaire ? Y a-t-il des couches cachées ? Comment assembler les fragments ?}

\section{Règles du Monde}
\textit{Les lois fondamentales de la physique et de la magie dans Rusttale. Gravité, entropie, gestion des ressources et limitations du moteur.}

\section{Modèle de Conscience}
\textit{Comment la conscience est-elle définie ? La différence entre les PNJ, le Joueur et le Protagoniste. La mécanique de l'"Âme".}

\section{Architecture Système}
\textit{La méta-couche. La simulation, le code, les bugs. Comment les contraintes techniques de Rust et Tetra influencent l'histoire.}

\section{Entités}
\textit{Profils des personnages clés, ennemis et anomalies. Leurs origines, fonctions et relations.}

\section{Position du Joueur}
\textit{Qui est le joueur ? Un observateur ? Un dieu ? Un virus ? La relation entre l'entrée (clavier/souris) et le résultat.}

\section{Modèle Temporel}
\textit{Linéarité vs Boucles. Sauvegardes, réinitialisations et persistance de la mémoire à travers les chronologies.}

\section{Événements Canoniques}
\textit{Les points fixes de l'histoire qui doivent se produire. Les vérités immuables de la chronologie.}

\section{Événements Manquants}
\textit{Données corrompues. Événements référencés mais supprimés ou inaccessibles. Les vides dans l'histoire.}

\section{Fragments Narratifs}
\textit{Nouvelles, lettres, journaux. Morceaux déconnectés de l'histoire qui construisent l'atmosphère.}

\section{Dialogues}
\textit{Conversations clés entre entités. Transcriptions des moments critiques.}

\section{Monologue Intérieur}
\textit{Les pensées intérieures du protagoniste. La lutte contre le code ou le contrôle du joueur.}

\section{États Cognitifs}
\textit{Niveaux de santé mentale. La corruption de l'esprit. Comment l'environnement affecte la psyché.}

\section{Modèle de Décision}
\textit{Chemins ramifiés. L'illusion du choix vs l'agence réelle. Les conséquences des actions.}

\section{Modes d'Échec}
\textit{États de Game Over. Bugs qui arrêtent la progression. Les équivalents de l'"Écran Bleu de la Mort" dans l'histoire.}

\section{Narration Non Fiable}
\textit{Contradictions dans l'histoire. Pourquoi le narrateur pourrait mentir ou se tromper.}

\section{Questions Ouvertes}
\textit{Mystères laissés intentionnellement non résolus. Carburant pour les théories de la communauté.}

\section{État Final}
\textit{Les conclusions possibles. La "Vraie Fin", la "Mauvaise Fin" et la "Fin Glitch".}

\section{Glossaire}
\textit{Définitions des termes spécifiques utilisés dans l'univers Rusttale.}

\section{Notes du Développeur}
\textit{Méta-commentaires des développeurs. Défis techniques devenus des fonctionnalités de l'histoire.}

\section{Fin du Livre}
\textit{La déclaration finale de clôture. Un au revoir ou un "À bientôt dans la prochaine boucle".}
