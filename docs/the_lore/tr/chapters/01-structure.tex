\chapter{Yapı \& Yönergeler}

\section{Üstveri}
\textit{Bu hikaye iterasyonunun sürümünü, tarihini ve benzersiz hash değerini tanımlayın. Bu, evrenin "build"ini oluşturur.}

\section{Yazar Notu}
\textit{Yaratıcıdan kişisel bir mesaj. Niyetler, temalar hakkında uyarılar veya tasarımın arkasındaki felsefe.}

\section{Okuma Rehberi}
\textit{Bu belgede nasıl gezineceğinize dair talimatlar. Doğrusal mı? Gizli katmanlar var mı? Parçalar nasıl birleştirilmeli?}

\section{Dünya Kuralları}
\textit{Rusttale'deki temel fizik ve büyü yasaları. Yerçekimi, entropi, kaynak yönetimi ve motorun sınırlamaları.}

\section{Bilinç Modeli}
\textit{Bilinç nasıl tanımlanır? NPC'ler, Oyuncu ve Protagonist arasındaki fark. "Ruh" mekaniği.}

\section{Sistem Mimarisi}
\textit{Meta-katman. Simülasyon, kod, hatalar (glitch). Rust ve Tetra'nın teknik kısıtlamalarının hikayeyi nasıl etkilediği.}

\section{Varlıklar}
\textit{Ana karakterlerin, düşmanların ve anomalilerin profilleri. Kökenleri, işlevleri ve ilişkileri.}

\section{Oyuncu Konumu}
\textit{Oyuncu kimdir? Bir gözlemci mi? Bir tanrı mı? Bir virüs mü? Girdi (klavye/fare) ile sonuç arasındaki ilişki.}

\section{Zaman Modeli}
\textit{Doğrusallık vs. Döngüler. Kayıt durumları, sıfırlamalar ve zaman çizelgeleri boyunca hafızanın kalıcılığı.}

\section{Kanon Olaylar}
\textit{Tarihte gerçekleşmesi zorunlu olan sabit noktalar. Zaman çizelgesinin değişmez gerçekleri.}

\section{Kayıp Olaylar}
\textit{Bozuk veriler. Referans verilen ancak silinmiş veya erişilemeyen olaylar. Tarihteki boşluklar.}

\section{Anlatı Parçaları}
\textit{Kısa hikayeler, mektuplar, günlükler. Atmosferi oluşturan kopuk hikaye parçaları.}

\section{Diyaloglar}
\textit{Varlıklar arasındaki kilit konuşmalar. Kritik anların dökümleri.}

\section{İç Monolog}
\textit{Protagonistin iç düşünceleri. Koda veya oyuncunun kontrolüne karşı verilen mücadele.}

\section{Bilişsel Durumlar}
\textit{Akıl sağlığı seviyeleri. Zihnin bozulması. Çevrenin psikolojiyi nasıl etkilediği.}

\section{Karar Modeli}
\textit{Dallanan yollar. Seçim illüzyonu vs. gerçek irade. Eylemlerin sonuçları.}

\section{Başarısızlık Modları}
\textit{Oyun sonu (Game Over) durumları. İlerlemeyi durduran hatalar. Hikayedeki "Mavi Ekran" karşılıkları.}

\section{Güvenilmez Anlatı}
\textit{Hikayedeki çelişkiler. Anlatıcı neden yalan söylüyor veya yanılıyor olabilir.}

\section{Açık Sorular}
\textit{Kasıtlı olarak çözümsüz bırakılan gizemler. Topluluk teorileri için yakıt.}

\section{Bitiş Durumu}
\textit{Olası sonuçlar. "Gerçek Son", "Kötü Son" ve "Hata (Glitch) Sonu".}

\section{Sözlük}
\textit{Rusttale evreninde kullanılan özel terimlerin tanımları.}

\section{Geliştirici Notları}
\textit{Geliştiricilerden meta-yorumlar. Hikaye özelliklerine dönüşen teknik zorluklar.}

\section{Kitap Sonu}
\textit{Son kapanış ifadesi. Bir veda veya "Bir sonraki döngüde görüşürüz".}
